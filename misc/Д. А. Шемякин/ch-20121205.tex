Германия своим рывком нарушила довольно устойчивое торговое равновесие Европы.
И речь идёт не столько о товаропроизводстве, сколько о тарифных новациях молодой
империи.

С середины 19 века торговые соглашения строились в Европе на условии
двухсторонних компромиссов, причём прей\-мущества, оказанные третьему
государству распространялись на договаривающиеся стороны \enquote{немедленно и
безвозмездно}.

Это была эра господства фритрейдерства. До 80х годов 19 века Россия
руководствовалась в отношениях с Западной и Центральной Европой т.н. автономным
тарифом, базирующимся на равенстве условий: единая пошлина на импортируемые
товары.

Для России с её \enquote{хлебной иглой}, на которой она прочно сидела столетиями,
настоящая война в экономике началась с введения Германией
аграрно -- протекционистских мер. Германия начала впервые в новейшей истории
практику искусственного устранения конкурентов с аграрного рынка. Лозунгом
аграрного немецкого протекционизма была защита национального
сельхозпроизводства. Целью же была не столько защита национальных интересов,
а наступление, подчинение и подавление экономики слабейших соседей -- конкурентов.

Когда Германия начала свою тарифную политику, конечно, она усилила
экономическую напряженность на континенте. Протекционисткие меры были введены,
вслед за Германией, во Франции, Италии, России. В активное движение пришли
страны, которые ранее не рассматривались как самостоятельные активные игроки на
европейском рынке. Эти \enquote{нейтральные страны} стали активно использовать
плоды от фиктивного импорта -- экспорта продукции экономически воюющих стран.

Что получила Россия от германских тарифных новаций? В системе русского экспорта
94,4 процента составляли продукты сельского хозяйства, промышленные изделия
составляли --- 3,5 процента, полуфабрикаты --- 2,1 процента. На хлебный экспорт
России уже давили США, Мексика, Аргентина, подключалась Канада. Всё это
заокеанское зерно стало поступать на европейские рынки, составляя конкуренцию
русскому хлебушку. И тут, в разгар заокеанской зерновой экспансии, Германия,
которая являлась крупнейшим клиентом российского аграрного сектора (50 процентов
зернового импорта Германии составляло русское зерно, соответственно Германия
потребляла 30 процентов русского хлебного экспорта), начинает строго
адресованную против России тарифную войну. Войну, в которой нам было очень
трудно выстоять и потому, что треть приходной части бюджета Российской империи
приходилась на поступления от сельхозвывоза, и потому, что нам было мало что
противопоставить германскому вызову с альтернативой технологических закупок.

Напомню, что все германские игры с тарифами начались при Бисмарке, который,
как уверены многие, был активным стороннником союза с Россией и чуть ли не
заклинал от конфликта с нами.

В чём выражалась германская стратегия выноса России с европейского рынка?

\begin{enumerate}
\item Немцы ставили во главу угла свои сугубо практические цели --- создание
самодавлеющей экономики, способной существовать в автономном режиме в случае
экстраординарных обстоятельств достаточно долгое время. Помимо развития
промышленности, скачками стало двигаться вперёд германское сельское хозяйство.
С-х Германии развивалось как за счёт экстенсивного расширения площадей, так и за
счёт интенсивной рационализации аграрного сектора.

\item С-х Германии стало переориентироваться на внешний рынок, вытеснение с
внутреннего рынка импорта и наступление на экономики потенциальных противников.
В 1879 году Германия впервые искусственно начала регулировать свой импорт,
обложив ввоз иностранного зерна и прочих сельхоз продуктов таможенной пошлиной
в 1 марку за 100 кг. Одновременно с этим Германия начала выкупать частные
железные дороги у их хозяев в госсобственность. Целей у этой национализации
было множство, одна из главнейших --- возможность диктовать тарифы перевозок
без согласования с кем бы то ни было.

\item Проценты от таможенных сборов шли немецким сельхозпроизводятелям для
поощрения производства.

\item Для производителей и экспортных фирм правительство Германии ввело
\enquote{вывозные премии}.
\end{enumerate}

Т.е. этап номер один. Повышаем пошлины на иностранное зерно, увеличиваем
производство собственного, вывозим, пользуясь обстоятельствами, своё зерно на
продажу (пока остальные не успели ввести у себя такой же протекционисткий
парадиз), средства вкладываем в дальнейшее удешевление собственной продукции
аграрного сектора. Россия, как основной поставщик зерна на германский рынок,
платит повышенные пошлины, деваться им, русским некуда, теряет в доходной части
бюджета и впадает в определенную зависимость от экономических решений
берлинского кабинета. Русские не вводят протекционисткие пошлины на германский
ввоз машиностроительной продукции за неимением альтернативы. У русских
индустриализация, им машины, станки, оптика и пр. очень нужны, а за свой хлеб
они будут просто меньше получать, вкладываясь косвенно в развитие германского
сельхозпроизводства.

В конце 70х у России положение было практически безвыходное. Хлеб надо было
вывозить любой ценой, бюджет трещал. Но Петербург ответил Берлину началом
взимания торговых пошлин в золотой валюте. Таможенные сборы (а ввозили к нам
необходимые промышленные товары, в которых российская экономика нуждалась
сильно) возросли с 25 процентов до 48 процентов. Естественно, что благоприятно
на темпы индустриализации России это сказаться не могло, тем более, что позже
России пришлось увеличивать пошлины сначала на 10 процентов, потом на 20
процентов по 108 статьям, потом ещё на 20 процентов по тем же статьям.

\vspace{1ex}\noindent
И тут уже начинается этап номер два.

\begin{enumerate}
\item В 1885 году германские пошлины на ввозимое зерно были повышены в три раза.
А через два года --- в пять раз. Россия ответила знаменитым покровительственным
тарифом 1891 года. Общая сумма таможенных повышений на весь импорт возросла с
14,7 процентов в 1877 до 32,7 процентов к 1892 году. К началу 90х годов
Российская империя уже могла себе позволить выпуск \enquote{негерманской} промышленной
продукции, опираясь на свои силы и силы негерманского капитала. Хотя, повторю,
20 процентов инвестиций в русскую экономику были все же немецкими.

Одним из наиболее ярких авторов протекционисткого тарифа 1891 года был
Д. И. Менделеев. Кто из нас не зачитывался его фундаментальным трудом
\enquote{Тарифный сбор или исследование о развитии промышленности в России в связи с
её общим таможенным тарифом 1891 г.}? С помощью цитат из этого труда было
разбито немало девичьих сердец.

В своих \enquote{Заветных мыслях} Дмитрий Иванович записал незабываемое:
\enquote{Существование государства, особенно его сила и движение верёд, при условии
значительных размеров страны и её населённости, немыслимы в обычных условиях
без внутренней обеспеченности в производстве необходимейших товаров, не только
потому, что в первой войне это скажется с великою силою, но и потому, что
недостаточное развитие внутреннего производства необходимейших товаров\ldots
отнимает от жителей много условий возможности правильного роста богатства
народного и ставит страну в тяжёлую зависимость от поставщиков этих необходимых
товаров}. Что здесь сказать? Стиль, близкий к Сумарокову и Хераскову, а
содержание отчаянное. Россия --- есть осаждённый лагерь, будем делать в этом
лагере всё сами, чтобы было чем от неприятеля отбиться. И это не какая-то
издевка с моей стороны. Автаркия как способ существования (пусть автаркия и в
сглаженной форме) --- это путь очень жёсткой индустриализации, я бы сказал,
жесточайшей по отношению и к материальным ресурсам страны, и к её обитателям с
их нематериальными устремлениями. Наша страна стала заложником собственного
могущества и даже  величия, имеющего в экономическом фундаменте комплекс:
основательное, но дико отсталое сельское хозяйство, отсуствие подоходного
налога и, следовательно, массу налогов косвенных, огромный военный бюджет и
потребность в ускоренном промышленном развитии в недружелюбном окружении.
Добавим к этому нерешённый аграрный вопрос и социальную напряжённоксть как в
деревне, так и в индустриализирующихся центрах. Эсперимент величайшей сложности
разворачивался, тяжело лавируя между войнами, революциями и архаикой
псевдо -- дворянского управления.

Менделеева привлекли к разработке таможенного тарифа почти случайно (это
особенность наша неизбывная, её обсуждать не будем: в Сбербанк набирают методом
перебора близких знакомых, в Ашан поманивают пахлавой и накрывают таджиков
сеткой, Менделеев просто зашёл в гости). \enquote{В сентябре 1889 года заехал
по-товарищески к И. А. Вышнеградскому, тогда министру финансов, чтобы поговорить
по нефтяным делам (Дмитрий Иванович умел глядеть в будущее, согласитесь:
заехал по-товарищески к министру финансов поговорить просто о нефтяных
делах\ldots Сколько бы сейчас людей согласилось оказаться на месте Дмитрия
Ивановича, чтобы по-товарищески так, по простому, заехать да и поговорить про
нефтяные дела хоть бы и к министру финансов)\ldots И он предложил мне заняться
таможенным тарифом по химическим продуктам и сделал меня членом совета торговли
и мануфактур\ldots} --- писал впоследствии предприимчивый учёный и общественный
деятель. Читаем далее: \enquote{Живо я принялся за дело, овладел им и напечатал этот
доклад (доклад о таможенных сборах, не имеющий прямого отношения ни к
мануфактурам, ни к химии) к рождеству\ldots} Дальше Менделеев несколько
скромничает: \enquote{Этим докладом определилось многое в дальнейшем ходе как всей моей
жизни, так и в направлении обсуждения тарифа, потому что цельность плана была
только тут (т.е. только в докладе Менделеева)\ldots} И сразу к Дмитрию
Ивановичу потянулись всякие единомышленники, о которых он пишет скромно, но
достойно: \enquote{С. Ю. Витте сразу стал моим союзником, за ним перешли многие
другие}.

Как вам сказать, повышение таможенного тарифа в таких объёмах --- это не
предложение дружбы. Это, если не начало экономической войны, то ультиматум,
требующий от Германии уступок.

\item План русского таможенного возмездия начал реализовываться. Мы надеялись,
что Германия пойдёт на уступки, мы считали такое повышение тарифов временным,
мы очень зависели от германского фактора, который во многом определял наше
положение на мировом рынке. Сравнительные цифры я уже приводил --- там всё
понятно без слов.

\item Германия на уступки не идёт, Вышнеградский пишет царю \enquote{В товарообмене
между Россией и Германией все преимущества находятся на стороне последней}.
Меры, проводимые Германией: \enquote{исключительно колебавшие доверие к нашему
финансовому положению, повлекли за собой падение вексельного курса}.

\item Потерпев неудачу в экономическом давлении на Германию, Россия в год
принятия протекционисткого тарифа вступает в военный союз с Францией.
\end{enumerate}

\vspace{1ex}\noindent
Этап номер три.

\begin{enumerate}
\item Германский ответ на русские тарифные демонстрации последовал мгновенно.
1 февраля 1892 года вступили в силу договоры Германии с Австро-Венгрией,
Бельгией, Италией, Швейцарией. Этим странам Германия пошлины на зерно снизила.
Швейцарии снизила, России --- нет.

\item Аналогичное понижение пошлин (а некоторым странам вообще было
предоставлено право беспошлинного экспорта) Германия допустила для Швеции,
Норвегии, Дании, Нидерландов, Греции, Турции, Мексики, Аргентины. И внимание!
Внимание! Снижение пошлин было сделано для Англии, Франции, США, Сербии. Но не
для России. Таможенные льготы получили Испания и Румыния! Но не Россия!
Колониальные владения Франции, Испании, Потругалии, Голландии, Бельгии получили
немецкие льготы на ввоз сельхозсырья. Конго получила, Индонезия, Кюрасао,
Вьетнам! Но не Россия!

\item Прижав к груди доклад Менделеева, Россия оказалась вообще вне германских
внешнеэкономических соглашений. Между Германией и Россией вообще не стало
никакого торгового договора.
\end{enumerate}

Вот начало русско -- германской войны.

Что называется, фактической, холодной, бескровной и беспощадной.

\vspace{1ex}\noindent
Этап номер четыре.

\begin{enumerate}
\item Германия повышает пошлины на ввоз русского зерна ещё раз. Теперь пошлина
равняется, по даным Минфина России, 100 процентам стоимости русского хлеба в
местах его производства.

\item Россия полностью была выбита с германского продовольственного рынка как
серьёзный игрок. В 1893 году вывоз ржи упал с 50562 тыс. пудов (1891 год) до
13656 тыс. пудов. За один год экспорт русской пшеницы упал с 54318 тыс. пудов до
42210 тыс. пудов.

Рынок Германии поделили США, Аргентина, Румыния, Сербия и Болгария. Две
заокеанские фантасмогории, два потенциальных сателлита и Сербия. Т.е. две
заведомо нейтральные страны (в случае войны), две фактические союзницы и Сербия.
Свой план автономного снабжения продовольствием, вытеснения потенциального
противника и т.п. Германия выполнила полостью и в сжатые сроки.

\item После этого всего, бои приняли затяжной характер. Россиия закрыла для
германских товаров Финляндию, Россия увеличила ластовый сбор с германских
судов за причал в русских гаваней - с 5 копеек до рубля. Немцы подняли
таможенные пошлины на русские товары (все русские товары) до 50 процентов их
стоимости.
\end{enumerate}

Всё сгладила несколько русско -- немецкая торговая конвенция и русско -- германский
торговый договор, заключённый сразу после поражения России в войне с Японией.
Уступки со стороны Германии были минимальны. Через год после русско -- германского
торгового соглашения, Россия скрепя сердцем вступила в союзнические отношения с
Великобританией. Её утолкали. Недомодернизированную, с гигантскими проблемами,
с неэффективной системой управления, её запихнули на весь этот \enquote{Титаник},
в третий класс, без шансов.

Мирное сосуществование на равноправных условиях Германской и Русской империй
оказалось невозможным. Времена такие пришли, что суперхищники уничтожали
хищников, прекрасная эпоха 1875 -- 1914 годов заканчивалась навсегда.

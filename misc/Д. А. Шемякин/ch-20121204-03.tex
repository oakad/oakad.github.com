Австрийский ультиматум Сербии. У нас его характеризуют как страшный и
унизительный, просто кошмарный и развязный гоп-окрик.

Ультиматум австрийцы написали довольно быстро, но вот одобрить его никак не
могли. Венгерские аристократы в политике вообще не хотели слышать ни про какую
войну. Во главе антивоенного выступления встал премьер -- министр Венгрии граф
Тиса. На Коронном Совете Тиса открыто высказался против войны и против захвата
сербских территорий. После давления со стороны императора вроде как согласился
с небольшой такой войной и был привлечён к составлению этого самого ультиматума.
Естественно, что антивоенные венгры тормозили утверждение ультиматума и смягчали
его до предела. Из Берлина стали раздаваться недоумённые крики, мол, вы чего
там?! Так всё удачно складывается. Из Петербурга только что отплыли президент
Французской Республики Пуакаре, премьер -- министр и министр иностранных дел
Вивиани, связи с Петербургом у них пока нет. Договорённости, как Франция сможет
помочь России на Балканах тоже нет! Мы на Николая давим, Николай мечется! Чего
вы, австрийцы ждёте?! Бейте!

И вот 23 июля 1914 года посланник Австро -- Венгрии барон Гизль вручил сербскому
правительству ультиматум. Так получилось, что в Белграде не оказалось никого из
великих сербских политиков. Такая вот дивная случайность. Старенький король
Петр I был не у дел, парламент распущен, премьер Сербии Пашич был в поездке,
министры в отпусках (война же на носу!), руководители сербской армии отдыхали
тоже. Где отдыхали? Сербский генералитет отдыхал в Австрии, в стране, которая
уже явно готовится к вторжению в их страну. Сербская армия не была мобилизована
даже на уровне полка, а была отправлена на полевые работы. Ультиматум по
сербской привычке встретили по-разному. Чиновники, например, начали спешную
эвакуацию из столицы. Успели ухватить за фалды  вице-премьера --- надо же
кому-то ультиматум вручать. Вице-премьер стал собирать по дачам правительство,
принц -- регент Александр написал два письма --- первое дяде своему --- союзнику
Германии и Австро -- Венгрии --- королю Италии, второе письмо решил всё ж написать
русскому царю.

При таком положении сербских дел стало очевидно, что страшенный ультиматум будет
сербами принят. Если бы Сербия приняла ультиматум Австрии, а к этому всё и шло,
то Россия теряла Балканы насовсем. В 1878 году мы отдали Австрии Боснию и
Герцеговину за невмешательство в русско -- турецкую войну, Австрия стала
обладательницей 5 млн. славянского населения, Болгария доходила до полного
разрыва диломатических отношений с Россией и демонстрировала свою подчёркнутую
лояльность Берлину, оставалась у России только Сербия и оперетта под названием
Черногория. Если Сербия приняла бы австрийские условия (в любой их трактовке),
то на 150-летнем присутствии России на Балканах можно было бы ставить крест,
забыть про вожделенные проливы и пр. На всю идеологию сверхдержавы можно было
бы махнуть рукой.

Авторитет сверхдержавной России стал спасать министр иностранных дел империи
Сазонов, который как и его недавний шеф П. П. Столыпин, сам бы отчаянным
германофилом.

Что могла селать Россия, сохраняя своё лицо, при неясной позиции союзников и
странной позиции Сербии? Союзники не радовали: британский министр иностранных
дел Грей сказал прямо: \enquote{Правительство Его Величества не хочет обсуждать
тему, кто прав: Автрия или Сербия}. Франция вела себя приличнее, но тоже ничем
серьёзным помочь России тогда не могла, а напротив, рассчитывала на русскую
помощь.

Что было делать министру Сазонову? Читать австрийский ультиматум и максимально
тянуть время.

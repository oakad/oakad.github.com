Сам император Франц -- Иосиф войны не хотел. К наследнику тёплых чувств не
испытывал, понимал, во что всё это безобразие на Балканах может вылиться для его
страны. И император делал очень многое, чтобы обуздать \enquote{военную партию},
которая хотела тут же начать стирать Сербию в порошок.

Его главным оппонентом был министр двора и общеимперский министр иностранных дел
Леопольд фон Бертхольд. Его главная цель была не столько начать войну Австрии и
Сербии, сколько привлечь на свою сторону не до конца уверенную в необходимости
войны Германию. План Бертхольда был удачно составлен: разбить Сербию в такие
короткие сроки, чтобы Россия не успела даже сдвинуться, парализованная мирными
инициативами Германии, а когда дело будет сделано, то оставалось совсем немного:
режим Николая II мог бы не выдержать \enquote{политическую Цусиму} на Балканах,
Германия могла бы резко изменить свой тон в отношении ослабленного политического
режима России, без российского фактора Франция, скованная возможным отпадением
от союза России, меняет вынужденно свою агрессивную политику. Германия диктует
условия Франции, которые Францию (без России и при колеблющемся Альбионе) может
и принять.

Собственно, план Бертхольда был прекрасен во всём. В случае успеха его
реализции, мировая война отодвинулась бы на некоторый срок, единственной жертвой
была бы Сербия и государь Николай Александрович, которого бы, скорее всего,
родственники затравили до вероятного отречения.

Но подвела очевидная неготовность автрийской армии к немедленному удару по
Белграду. Да и Германия заявила, что для неё первоочередной задачей является
вопрос с Францией и Берлину не до балканских мелочей. В Берлине вообще
сараевские события не произвели никакого серьёзного вечатления. Во Франции ---
тоже. Англия всё тянула. И внезапно Россия оказалась перед чудесной перспективой
--- начинать войну с Австрией в одиночку. Из-за Сербии.

Император Франц -- Иосиф обращается, предчувствуя такое развитие событий, в Берлин
к кайзеру с письмом от 5 июля, в котором чёрным по белому сказано: вина Сербии
доказана быть не может в сараевском случае, но по существу нельзя сомневаться
в том, что политика сербского правительства очевидно враждебна как Австрии, так
и Германии и мешает вовлечению в союз Турции и Болгарии. Письмо Франца -- Иосифа
натолкнуло кайзера на идею разделения войны. Быстрый разгром Сербии,
присоединение к Тройственному союзу всякой полезной мелочи: Румынии, Болгарии,
Турции --- как итог перевод вектора российской обороны на юго -- западное
направление (при условии, что Одесский военный округ, противостоящий по планам
генштаба румынскому вторжению, будет не способен угрожать южному флангу
авcтро-венгров, а Болгария сделает невозможной реализацию планов русского
десанта на турецкое побережье).

Вильгельм пишет на донесении своего посла в Вене фон Чиршки: \enquote{Теперь или
никогда}.

Франция только обещает России какую-то военную помощь в случае
сербско -- русско -- австрийской войны, Альбион молчит. Русскому командованию в
эти дни очень трудно понять, что вообще происходит. Где тут наше \enquote{Сердечное
согласие}? Русской разведке становится известно, что Вильгельм дал ответ
австрийскому послу графу Сегени начинать войну с Сербией немедленно и обещал,
что Германия \enquote{с обычной своей союзнической верностью поддержит Автро -- Венгрию!}
Ни слова про вторжение во Францию!

И русская военная и внешнеполитическая элита, влючая государя нашего обожаемого,
начала паниковать.

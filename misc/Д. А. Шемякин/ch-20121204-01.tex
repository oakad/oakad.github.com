С точки зрения политики Россия вступила в войну в странном для себя положении
\enquote{привлечённой звезды}. Вытряхнули из мешка на корпоративе.

Убийство эрц-герцога Фердинанда не могло служить поводом для начала войны,
поводом для начала войны на Востоке Европы было шоу, устроенное всеми сторонами:
Австро-Венгрией, Германией, Сербией и Россией.

Наследника австро-венгерского престола убили на территории Австро -- Венгрии,
убийцами были боснийцы --- подданные Австро -- Венгрии, и \enquote{сербский след}
был обнаружен значительно позднее начала самой войны. Т.е. убили автрийского
наследника на его наследуемой территории, его же потенциальные подданные.

Вслед за убийствои Фердинанда начался спектакль. Европа жила тогда традициями
предварительных объявлений войн, причём требовалось подробно объяснить не только
коварные замыслы врага, но и подчеркнуть все свои действия по предотвращению
войны, определить в самом благородном свете свои цели и методы. Очень это было
всё хлопотно.

Вдобавок ко всему, все участницы будущей войны не хотели её в 1914, а хотели в
1915. Не всё в порядке было в союзнических лагерях, что у Антанты, что в
Тройственном Союзе. В каждом лагере был потенциальный предатель, который мог
соскочить в любой момент, а то и переметнуться на другую сторону. Наконец, нужно
было время для вербовки малозначительных в мирное время, но важных в военное
время \enquote{второстепенных участников}. И наконец, самый главный вопрос ---
вопрос о многомиллионной мобилизации армий требовал значительного времени. А тут
лето в зените, пока то, пока сё\ldots{} А там оcень --- зима. Не хорошее время для
войны.

Июльский кризис начался не в Сараево, а в Вене. Убийство Фердинанда расстроило
далеко не всех, венгров оно просто обрадовало, но общая атмосфера в Австрии была
далека от благостной. На арену вышла армия.

С 1815 года мир был естественным каркасом европейской жизни, с 1815 года не было
ни одной европейской войны с участием всех крупных европейских государств. С
1871 года ни одна европейская армия не открывала огонь по другой европейской
армии. Конечно, мы не учитываем балканские страны с их полуевропейским
положением.

Война для европейца к 1914 году была делом историческим, почти несерьёзным,
немного романтичным и даже красивым. Служба в призывной армии была процессом
инициации, в США и Англии призывной армии вообще не существовало. Для офицеров
армия была местом работы, при отсутствии врага работа была невесёлой, рутинной
и не особо перспективной в карьерном росте и денежном плане. В основном, армии
использовались на колониальных перефериях, в войнах в которых риски были скорее
медицинского харктера, нежели военного. В американо -- испанскую войну 1898 года
из 274000 американских солдат 379 человек было убито, 1600 ранено и 5000 погибло
от тропических болезней. Франция теряла в среднем по 8 офицеров в год в своих
колониальных операциях (не счиатая Тонкина, где было потеряна половина из 300
офицеров, погибших в период с 1871 по 1903 гг. Наиболее серьёзные потери понесла
Британия в англо -- бурскую войну --- отправив в Южную Африку 450000 человек за
три года англичане потеряли 29000 убитыми и 16000 померших от болезней.

При всём этом труд солдата и офицера был более безопасен, чем труд торгового
моряка или шахтёра, или рабочего сталилитейного завода. За три года,
предшествующих войне, в Англии каждый год погибало примерно 1430 шахтёров, а
165000 получали увечья разной степени тяжести. Это 10 процентов рабочей силы
горнорудной промышлености империи.

Жизнь солдата и офицера европейской армии была спокойна и несколько скучна. А
тут такое! Убили наследника! Это же очень перспективно для армии!

Говоря в целом, убийство Фердинанда не тянуло на повод лля мировой войны никак.

Убийство осложнило дипломатические отношения, но эти осложнения не делали войну
автоматически неизбежной в 1914 году. Противоречия Германии и Франции за Эльзас
и Лотарингию (а эти области стоили полномасшабной войны) абсолютно не волновали
не только союзников Франции (Англию и Росссию), но были совершенно безразличны
для Австрии. Борьба Австрии и России за влияние на Балканах не тревожили
политиков Германии, не говоря уже про Альбион. У Франции не было вообще
претензий к Австрии, а у России претензии к Германии носили совершенно
экономический и регулируемый характер.

То, что война будет и будет мировой решила, возможно того и не желая Англия.

Когда накунуне 1914 года (в период с 1903 по 1907) Британия решилась
присоединиться с оговорками к антигерманскому пакту, это решительно радостно
взбудоражила и Францию, и Россию. Одна из причин подъёма антигерманской риторики
в Париже и Санкт-Петербурге --- это то, что одна только Британия к 1914 году
обеспечивала 44 процента мировых капиталовложения. А Франция, Германия, США,
Бельгия, Голландия, Швейцария и пр., натужась, выдавали сообща оставшиеся 55
процентов мировых капвложений. От 50 до 25 процентов экспорта всех стран
Латинской Америки, Азии и Африки поступали в одну  Великобританию, на все страны
Европы (вместе взятые) приходилось 30 процентов. Операции на биржах Сити только
за счёт стоимости торговых и финансовых услуг давали 142 млн. фунтов стерлингов
в год, что покрывало торговый дефицит Британии практически полностью. Фунт был
мировой валютой. И некоторый упадок британской промышленности (в сранении с
темпами Германии и США), внимание, только усиливал финансовые позиции и общее
благосостояние Британии. Механизм был простой: бешенно развивающие свои
промышленности станы покупали всё больше и больше профилирующих товаров в
странах колониального и полуколониального типа (подконтрольных, по большей
части, Британской империи), транспортировка этих товаров из колониальной сферы
к центрам производства осуществлялась по большей части британским торговым
флотом, который на 12 процентов превосходил тоннаж флотов всех европейских
стран, при подавляющем господстве на океанских просторах флота Его Величества.
Флотские риски страховались на 78 процентов в Лондоне. Интенсивное потребление
ресурсов из стран-доноров увеличивало \enquote{зависимость} быстро растущих экономик,
германской, в первую очередь, от \enquote{зависимого мира}, зависимого от Британии.
А Британия в одиночку (я это подчеркну) восстанавливала мировой торговый баланс
через увеличение импорта из стран-соперников в свои \enquote{зависимые миры}, т.е. де
факто к себе импортируя, при этом избегая тарифных осложнений и наживаясь на
фрахте.

Колебательное согласие Великобритании громить Германию --- вот это был самый
главный повод для начала мировой войны (которую привычно называют Первой
Мировой, с чем я соглашусь вряд ли).
